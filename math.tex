\documentclass[12pt, a4paper]{article}
\usepackage[utf8]{inputenc}
\usepackage[T1]{fontenc}
\usepackage{pgfplots}
\usepackage{tikz}
\usepackage{amssymb}
\usepackage{amsmath}
\usepackage{amsfonts}
\usepackage{graphicx}

\pgfplotsset{compat=1.18, width=10cm}

\begin{document}
\tableofcontents
\newpage
\section{Precalculus}
\subsection{Algebra}
\subsection{Trigonometry}
\section{Calculus}
\section{Linear Algebra}
\subsection{Wektor}
Wektor to uporządkowana para liczb. Jeśli wektor ma początek to jest to, wektor
zaczepiony który jest oznaczany symbolem $\overrightarrow{AB}$. Jeżeli dane są punkty
$A = (x_1,y_1)$ oraz $B = (x_2,y_2)$,
to współrzędne wektora $\overrightarrow{AB}$ określa wzór: $$\overrightarrow{AB} = [x_2-x_1,y_2-y_1]$$
Jeśli natomiast wektor nie ma początku to jest to wektor swobodny który
jest oznaczany symbolem $\overrightarrow{v}, \overrightarrow{u}, \overrightarrow{w}$.
$$\overrightarrow{u} = \overrightarrow{w} \Longleftrightarrow u_x = w_x \wedge u_y = w_y$$
Na rysunku poniżej został przedstawiony wygląd wektora $[3,2]$ i $[-2,4]$ w układzie współrzędnych:

\begin{center}
\begin{tikzpicture}
\begin{axis}[xmin=-4.5,xmax=4.5,ymin=-2.5,ymax=4.5,axis lines=middle,
            xtick={-4,-3,...,4},ytick={-4,-3,...,4}, xlabel=$x$, ylabel=$y$,
            ]
  \addplot[domain=0:4, samples=250, ultra thick,blue, ->] coordinates {
    (0,0)
    (3,2)
  }
  node[pos=1.0, above left, blue]{$[3,2]$};
  \addplot[domain=0:4, samples=250,dashed] coordinates {
    (0,2)
    (3,2)
    (3,0)
  };

  \addplot[domain=0:4, samples=250, ultra thick,red, ->] coordinates {
    (0,0)
    (-2,4)
  }
  node[pos=1.0, below left, red]{$[-2,4]$};
  \addplot[domain=0:4, samples=250,dashed] coordinates {
    (0,4)
    (-2,4)
    (-2,0)
  };

\end{axis}
\end{tikzpicture}
\end{center}


\end{document}
